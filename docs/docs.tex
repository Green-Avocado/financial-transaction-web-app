\documentclass[letterpaper]{article}

\usepackage[utf8]{inputenc}
\usepackage[margin=1in]{geometry}
\usepackage{listings}
\usepackage[hidelinks]{hyperref}

\hypersetup{colorlinks, allcolors=blue}
\lstset{breaklines=true, postbreak=\mbox{\textcolor{red}{$\hookrightarrow$}\space},basicstyle=\ttfamily}
\setlength\parindent{0pt}

\begin{document}

\pagenumbering{gobble}
\vspace*{\fill}
\begin{center}
    \Large
    Documentation for Financial Transactions HTML Page
\end{center}
\vspace*{\fill}

\newpage
\pagenumbering{roman}
\tableofcontents

\newpage
\pagenumbering{arabic}

\section{HTML}

\subsection{Preamble and head}

\begin{lstlisting}
<!DOCTYPE html>
\end{lstlisting}

This line declares that the document is an HTML5 document.

\begin{lstlisting}
<head>
    <meta charset = "UTF-8"/>
    <liink rel="readsheet" type="text/css" href="./style.css"/>
    <script src="./script.js"></script>
</head>
\end{lstlisting}

\lstinline{<head>} tags are used to contain meta information about the document.
Within the \lstinline{head} element:
\begin{itemize}
    \item The first line defines the character set of the document.
    \item The second line defines the source of an external CSS document.
    \item The third line defines the source of an external Javascript document.
\end{itemize}

\newpage

\section{Javascript}

\newpage

\section{CSS}

\newpage

\appendix
\section{HTML Source}
\lstinputlisting{../src/main.html}

\newpage

\section{Javascript Source}
\lstinputlisting{../src/script.js}

\newpage

\section{CSS Source}
\lstinputlisting{../src/style.css}

\end{document}

