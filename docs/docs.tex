\documentclass[letterpaper]{article}

\usepackage[utf8]{inputenc}
\usepackage[margin=1in]{geometry}
\usepackage{listings}
\usepackage[hidelinks]{hyperref}

\hypersetup{colorlinks, allcolors=blue}
\lstset {
    breaklines=true,
    postbreak=\mbox{\textcolor{red}{$\hookrightarrow$}\space},
    basicstyle=\ttfamily,
    numbers=left,
    numberstyle=\normalsize,
    numbersep=10pt,
    frame=single,
}
\setlength\parindent{0pt}

\begin{document}

\pagenumbering{gobble}
\vspace*{\fill}
\begin{center}
    \Large
    Documentation for Financial Transactions HTML Page

    \large
    Jason N.

    April 26, 2020
\end{center}
\vspace*{\fill}

\newpage
\pagenumbering{roman}
\tableofcontents

\newpage
\pagenumbering{arabic}
\parskip 10pt

\section{Foreword}

Some of the code samples in this document were copied by hand.
If there are any discrepencies between code in this document and in the source files, refer to the source files.

This does not apply to the appendix.
Code in the appendix was generated directly from the source files.

\section{HTML}\label{HTML}

\subsection{Preamble and head}

This line declares that the document is an HTML5 document.

\begin{lstlisting}[firstnumber=1]
<!DOCTYPE html>
\end{lstlisting}

\lstinline{<head>} tags are used to contain meta information about the document.

\begin{lstlisting}[firstnumber=2]
<head>
    <meta charset = "UTF-8"/>
    <link rel="stylesheet" type="text/css" href="./style.css"/>
    <script src="./script.js"></script>
</head>
\end{lstlisting}

Within the \lstinline{head} element:

\begin{itemize}
    \item The first line defines the character set of the document.
    \item The second line defines the source of an external CSS document.
    \item The third line defines the source of an external Javascript document.
\end{itemize}

\subsection{Inputs}

The input section of this page is contained within \lstinline{<article>} tags for the purpose of organisation.
This can be used to facilitate styling this part of the page with CSS if desired.

\begin{lstlisting}[firstnumber=10]
<article id="inputFields">
\end{lstlisting}

The \lstinline{article} element has been assigned a unique id for the purpose of styling.
Specifically, this id is used to define padding and overflow.
This is described in further detail in section~\ref{overflow-x} of this document.

All input fields and buttons are contained within \lstinline{<form>} tags.
Althought this is not strictly necessary for the purpose of this project, it is useful for organising data and specifying the fields from which data should be submitted.

\begin{lstlisting}[firstnumber=11]
<form onsubmit="return false" autocomplete="off">
\end{lstlisting}

The attribute \lstinline{onsubmit} is used to define a Javascript function to be executed when pressed.
The form expects that \lstinline{true} is returned when data is successfully submitted.
If so, the default behaviour is to clear the fields and enter the data in the browser URL bar as arguments.
To prevent this behaviour, \lstinline{onsubmit} is set to \lstinline{return false}.

The attribute \lstinline{autocomplete} can be used to specify whether user input from a previous session should be used to populate input fields.
This attribute also determines whether or not suggestions are displayed when the user enters data.
In this case, \lstinline{autocomplete} has been set to \lstinline{off} to prevent these actions from occuring.
This does not affect the functionality of the program.

The buttons and input fields within the \lstinline{form} element are contained within \lstinline{<section>} tags for organisation.
This is primarily done to allow elements to be positioned properly by the CSS file.

\subsubsection{Common attributes}

All \lstinline{input} elements in this \lstinline{form} have been assigned a \lstinline{name} attribute.
The \lstinline{name} attribute is not strictly relevant in this case, but is often used to identify the data when submitting to a database.

All \lstinline{input} elements have the \lstinline{required} attribute.
Normally this prevents a \lstinline{form} from being submitted unless all \lstinline{required} fields contain data.
This does not apply to our case as we have disabled the built-in submit function.
However, it does still outline missing fields in red.

\subsubsection{Labels}

Each of the inputs are given a label to specify to a user the type of information which should be entered in the given field.
This is done with the \lstinline{input} element.

\begin{lstlisting}[firstnumber=12]
<label for="date">Date:</label><br/>
\end{lstlisting}

The \lstinline{for} attribute is used to specify an element which corresponds to this label.
This is done by setting the attribute to the id of the other element.
Labels allow a user to select an input field by clicking the label rather than the field itself.
Labels are also used to facilitate the use of assistive technologies.

\subsubsection{Date}

The date of a transaction is specified through the use of an \lstinline{input} element with a \lstinline{type} attribute of \lstinline{date}.
This can be used to effectively restrict the input to a valid date format and provides an intuitive method for inputting data.

\begin{lstlisting}[firstnumber=11]
<section>
    <label for="date">Date:</label><br/>
    <input id="date" name="date" type="date" required/>
</section>
\end{lstlisting}

This type of input field is also useful for interpreting dates in Javascript, as it provides methods which return the date in various formats to facilitate displaying and comparing dates.

\subsubsection{Text}

\lstinline{input} elements with a \lstinline{type} attribute of \lstinline{text} can be used to retrieve a string from a user.
This is also the field used for numbers, as these can be easily verified and converted in Javascript.

\begin{lstlisting}[firstnumber=16]
<section>
    <label for="account">Account Number:</label><br/>
    <input id="account" name="account" type="text" placeholder="Account Number" required/>
</section>
\end{lstlisting}

The advantage of taking numbers from an input field is that it allows for characters such as \$ to be included.
In the case of this project, users are able to submit Dollar Amounts as purely numberic values, or in a currency format.
Currently, the program only accepts dollars as a currency, however, it is possible to allow and store any number of currencies.
These characters, of course, have to be filtered out before the number is interpretted and re-inserted before displaying the value.

\subsubsection{List}

Dropdown lists are created using \lstinline{<select>} tags containing \lstinline{option} elements.
Each \lstinline{option} element represents a possible value, the first element is selected by default.

\begin{lstlisting}[firstnumber=21]
<section>
    <label for="type">Transaction Type:</label><br/>
    <select id="type" name="type">
        <option value=""></option>
        <option value="BUY">BUY</option>
        <option value="SELL">SELL</option>
        <option value="DIVIDEND">DIVIDEND</option>
        <option value="INTEREST">INTEREST</option>
        <option value="WITHDRAW">WITHDRAW</option>
        <option value="DEPOSIT">DEPOSIT</option>
    </select>
</section>
\end{lstlisting}

The \lstinline{innerHTML} of an \lstinline{option} element is the text that will be displayed to the user.
The \lstinline{value} attribute of the element is the value that will be read by Javascript.
For this project, the \lstinline{value} and \lstinline{innerHTML} were made to be identical so that the text in the table would be the same as the text the user had seen in the list.

\subsubsection{Buttons}

\lstinline{button} elements are clickable elements which can execute Javascript code specified by an \lstinline{onclick} attribute.
Text within the \lstinline{innerHTML} of the \lstinline{button} will be displayed as text within the button, which is useful for communicating the purpose of the button.

\begin{lstlisting}[firstnumber=49]
<section>
    <button id="add" type="submit" onclick="addTransactionButton();">Add Transaction</button>
    <button id="save" type="submit" hidden="true" onclick="saveChanges();">Save</button>
    <button id="discard" type="button" hidden="true" onclick="discardChanges();">Discard</button>
</section>
\end{lstlisting}

In this case, three buttons are present, each set to execute a different Javascript function when clicked.

Two of the three buttons have a \lstinline{type} attribute of \lstinline{submit}.
This causes each function to trigger the \lstinline{submit} event along with the Javascript function.
However, for this project, this event has been disabled by the \lstinline{form} \lstinline{onsubmit="return false"} attribute.
Thus, the only difference is that this causes missing fields to be outlined in red when the button is pressed.

The last button is of \lstinline{type} \lstinline{button}.
This element functions exactly the same, except it does not trigger the \lstinline{submit} event.
For this project, this means that missing fields will not be highlighted red, as this is not necessary for the `Discard Changes' button.

Two of the three buttons also have the \lstinline{hidden="true"} attribute.
This causes the page to render as if these elements did not exist, as these elements are only relevant when editing a row.
All three buttons are given unique ids so that \lstinline{hidden} attributes can be added or removed as needed.

\subsection{Table}

\subsubsection{thead}

The header of the table is enclosed in \lstinline{<thead>} tags.
This element includes the first row of the table, denoted by \lstinline{<tr>} tags, which contains headers for each column.

Every cell in the header is denoted by \lstinline{<td>} tags.
These cells differ from normal cells, such as those in the body of the table, in how they format their contents.
Using this element for header cells makes them stand out slightly as well as making it easier to differentiate when styling with CSS.

\begin{lstlisting}[firstnumber=61]
<th>
    <section>
        Transaction ID
    </section>
    <section class="sort">
        <button type="button" onclick="sortTable(0, true)">^</button>
        <button type="button" onclick="sortTable(0, false)">V</button>
    </section>
</th>
\end{lstlisting}

The first 8 header cells are split into two separate \lstinline{section} elements.
This was done to allow for the proper positioning of the header text and the sort buttons.
For this reason, the latter \lstinline{section} element is given the class \lstinline{sort} to differentiate between the two.

Each of the first 8 header cells contain two buttons for sorting.
All sorting buttons call the same function \lstinline{sort(column, ascending)}, however, they pass different arguments to this function.
The first argument is the column number, starting from 0, which allows the Javascript function to determine which column to use when comparing rows.
The second argument defines whether data should be sorted in ascending or descending order.

The last header cell contains nothing but text.
This column is used to contain the delete and edit buttons created for each row.

\begin{lstlisting}[firstnumber=133]
<th>Actions</th>
\end{lstlisting}

\subsubsection{tbody}

The table body is enclosed in \lstinline{<tbody>} tags.
This element is meant to be the main container of data in a table.

\begin{lstlisting}[firstnumber=136]
<tbody id="tableBody">
\end{lstlisting}

The table body is important for this project as it is the parent element of all data which will be manipulated.
For this reason, it has been given a unique id to reference in Javascript.
This was not strictly necessary, as it is also possible to reference this element by its tag name, being the only \lstinline{tbody} element.
Nevertheless, I consider this to be good practice as it is clear which element is being referred to in Javascript and allows for other tables to be added in the future if necessary without breaking the current functionality.

\newpage

\section{Javascript}\label{JS}

\subsection{getData()}

This function is used to retrieve and format data from the input fields.

\begin{lstlisting}[firstnumber=1]
function getData() {
    var date = document.getElementById("date");
    var account = document.getElementById("account").value;
    var type = document.getElementById("type").value;
    var security = document.getElementById("security").value;
    var amount = document.getElementById("amount").value;
    var dAmount = document.getElementById("dAmount").value;

    amount = Number(amount);

    if(dAmount[0] == '$') {
        dAmount = dAmount.substr(1);
    }
    dAmount = Number(dAmount);

    if(validate(date, account, type, security, amount, dAmount)) {
        var costBasis = calculateCostBasis(amount, dAmount);
        date = date.value;
        dAmount = '$' + dAmount.toFixed(2);

        return [ date, account, type, security, amount, dAmount, costBasis ];
    }
    else return false;
}
\end{lstlisting}

The function checks whether the data is valid by calling the \lstinline{validate()} function.
If so, data is formatted and sent to the function which called \lstinline{getData()}.
Currently, the caller is either \lstinline{addTransactionButton()} or \lstinline{saveChanges()}.

The function first stores the \lstinline{date} element and the raw values of the other input fields.
\lstinline{date} is treated differently as the element includes useful methods for comparing the date in different formats.

\begin{lstlisting}[firstnumber=2]
var date = document.getElementById("date");
var account = document.getElementById("account").value;
var type = document.getElementById("type").value;
var security = document.getElementById("security").value;
var amount = document.getElementById("amount").value;
var dAmount = document.getElementById("dAmount").value;
\end{lstlisting}

Next, some of the data is processed.
\lstinline{amount} is converted from a string, as it originated from a text field, to a number.
This is done with the built-in \lstinline{Number()} function, which takes a string as an argument and returns it as a numeric value when possible.
If the argument cannot be converted, the function returns \lstinline{NaN} or `Not a Number'.
We are not concerned with validating that the value can be converted at this stage, as we can simply check if the value is \lstinline{NaN} during the validation stage, therefore it is safe to convert to a number here.

\begin{lstlisting}[firstnumber=9]
amount = Number(amount);
\end{lstlisting}

A similar conversion is performed on the \lstinline{dAmount} value.
However, before this occurs, we check whether the first character in the string is a dollar sign.
If so, we remove the dollar sign by taking a substring of \lstinline{dAmount} which includes everything including and after the second character.
This effectively removes the dollar sign from the string, allowing it to be converted to a numeric value.

\begin{lstlisting}[firstnumber=9]
if(dAmount[0] == '$') {
    dAmount = dAmount.substr(1);
}
dAmount = Number(dAmount);
\end{lstlisting}

The function then calls \lstinline{validate} and passes all the stored variables as arguments to determine whether all the data is valid.
If not, the function will return \lstinline{false} and exit, preventing subsequent steps from occuring.

\begin{lstlisting}[firstnumber=16]
if(validate(date, account, type, security, amount, dAmount)) {
    var costBasis = calculateCostBasis(amount, dAmount);
    date = date.value;
    dAmount = '$' + dAmount.toFixed(2);

    return [ date, account, type, security, amount, dAmount, costBasis ];
}
else return false;
\end{lstlisting}

If all data is valid, the function calculates and stores the costBasis by calling \lstinline{calculateCostBasis()} and passing the necessary values.
The function also formats the date and dollar amount in the correct formats to be exported to the table.
Lastly, the function returns a list including all the data to the caller.

\subsection{validate()}

\begin{lstlisting}[firstnumber=26]
function validate(date, account, type, security, amount, dAmount) {
    if(!validateDate(date)) return false;
    if(!validateAccount(account)) return false;
    if(!validateType(type)) return false;
    if(!validateSecurity(security)) return false;
    if(!validateAmount(amount)) return false;
    if(!validateDAmount(dAmount)) return false;

    return true;
}
\end{lstlisting}

\subsection{generateId()}

\begin{lstlisting}[firstnumber=109]
function generateId() {
    var id = '';
    var idLength = 6;

    var characters = 'ABCDEFGHIJKLMNOPQRSTUVWXYZ0123456789';
    var charactersLength = characters.length;

    var unique = false;

    while(!unique) {
        for(var i = 0; i < idLength; i++) {
            id += characters.charAt(Math.floor(Math.random() * charactersLength));
        }

        unique = true;
        for(var i = 0; i < document.getElementsByClassName('idCell').length; i++) {
            if(document.getElementsByClassName('idCell')[i].innerText == id) {
                unique = false;
                break;
            }
        }
    }
    return id;
}
\end{lstlisting}

\subsection{calculateCostBasis()}

\begin{lstlisting}[firstnumber=134]
function calculateCostBasis(amount, dAmount) {
    costBasis = '$' + (dAmount / amount).toFixed(2);
    return costBasis;
}
\end{lstlisting}

\subsection{addTransaction()}

\begin{lstlisting}[firstnumber=139]
function addTransaction(id, date, account, type, security, amount, dAmount, costBasis) {
    var tableBody = document.getElementById('tableBody');
    var newRow = tableBody.insertRow(0);
    newRow.classList += "bodyRow";

    var actionsContent = "<button type='button' onclick='editRow(this)'>Edit</button> <button type='button' onclick='deleteRow(this)'>Delete</button>";
    var rowContents = [id, date, account, type, security, amount, dAmount, costBasis, actionsContent];

    for(var i = 0; i < rowContents.length; i++) {
        var newCell = newRow.insertCell(i);
        newCell.innerHTML = rowContents[i];
        if(i == 0) {
            newCell.classList += "idCell";
        }
    }
}
\end{lstlisting}

\subsection{deleteRow()}

\begin{lstlisting}[firstnumber=173]
function deleteRow(button) {
    var row = button.parentElement.parentElement;
    document.getElementById("tableBody").removeChild(row);

    if(document.getElementsByClassName('editing').length == 0) {
        document.getElementById('add').removeAttribute('hidden');
        document.getElementById('save').setAttribute('hidden', true);
        document.getElementById('discard').setAttribute('hidden', true);
    }
}
\end{lstlisting}

\subsection{editRow()}

\begin{lstlisting}[firstnumber=184]
function editRow(button) {
    if(document.getElementsByClassName('editing').length > 0)
        document.getElementsByClassName('editing')[0].classList = "bodyRow";

    var row = button.parentElement.parentElement;
    var rowContent = row.getElementsByTagName('td');
    row.classList = "bodyRow editing";

    document.getElementById('date').value = rowContent[1].innerText;
    document.getElementById('account').value = rowContent[2].innerText;
    document.getElementById('type').value = rowContent[3].innerText;
    document.getElementById('security').value = rowContent[4].innerText;
    document.getElementById('amount').value = rowContent[5].innerText;
    document.getElementById('dAmount').value = rowContent[6].innerText;

    document.getElementById('add').setAttribute('hidden', true);
    document.getElementById('save').removeAttribute('hidden');
    document.getElementById('discard').removeAttribute('hidden');
}
\end{lstlisting}

\subsection{saveChanges()}

\begin{lstlisting}[firstnumber=204]
function saveChanges() {
    data = getData();
    if(data) {
        rowToEdit = document.getElementsByClassName('editing')[0];
        cellsToEdit = rowToEdit.getElementsByTagName('td');

        for(var i = 0; i < data.length; i++) {
            cellsToEdit[i + 1].innerHTML = data[i];
        }
        rowToEdit.classList = "bodyRow";
    }

    document.getElementById('add').removeAttribute('hidden');
    document.getElementById('save').setAttribute('hidden', true);
    document.getElementById('discard').setAttribute('hidden', true);
}
\end{lstlisting}

\subsection{discardChanges()}

\begin{lstlisting}[firstnumber=221]
function discardChanges() {
    document.getElementsByClassName('editing')[0].classList = "bodyRow";

    document.getElementById('add').removeAttribute('hidden');
    document.getElementById('save').setAttribute('hidden', true);
    document.getElementById('discard').setAttribute('hidden', true);
}
\end{lstlisting}

\subsection{sortTable()}

\begin{lstlisting}[firstnumber=229]
function sortTable(column, ascending) {
    var tableBody = document.getElementById('tableBody');
    var rows = document.getElementsByClassName('bodyRow');

    var sorting = true;
    while(sorting) {
        sorting = false;
        for(var i = 0; i < (rows.length - 1); i++) {
            rowA = rows[i].getElementsByTagName('td')[column];
            rowB = rows[i + 1].getElementsByTagName('td')[column];

            var swap = false;

            if(ascending && rowA.innerHTML.toLowerCase() > rowB.innerHTML.toLowerCase()) swap = true;
            else if(!ascending && rowA.innerHTML.toLowerCase() < rowB.innerHTML.toLowerCase()) swap = true;

            if(swap) {
                sorting = true;
                rows[i].parentNode.insertBefore(rows[i + 1], rows[i]);
            }
        }
    }
}
\end{lstlisting}

\newpage

\section{CSS}\label{CSS}

\subsection{Vertical Scrolling Table}

\begin{lstlisting}[firstnumber=29]
#table {
    max-height: 80vh;
    overflow: auto;
}
\end{lstlisting}

\begin{lstlisting}[firstnumber=39]
th {
    min-width: 200px;
    width: 10%;
    position: sticky;
    background: white;
    top: 0;
}
\end{lstlisting}

\subsection{Horizontal Scrolling on Overflow}\label{overflow-x}

\begin{lstlisting}[firstnumber=5]
#inputFields {
    padding: 10px 0;
    overflow-x: auto;
}

form {
    min-width: 1900px;
}
\end{lstlisting}

\subsection{Miscellaneous}

\subsubsection{Sort buttons}

\begin{lstlisting}[firstnumber=47]
th > section {
    width: 80%;
    display: inline-block;
    padding: 0;
    margin: 0;
}

.sort {
    width: 10%;
}

.sort > button {
    padding: 0;
    border: 0;
    display: block;
    width: 100%;
}
\end{lstlisting}

\subsubsection{Editing highlight}

\begin{lstlisting}[firstnumber=65]
.editing {
    background-color: yellow;
}
\end{lstlisting}

\subsubsection{Table borders}

\begin{lstlisting}[firstnumber=69]
#table,
table,
td,
th {
    box-shadow: 1px 1px black, inset 1px 1px black;
}
\end{lstlisting}

\newpage

\appendix
\section{HTML Source Code}
\lstinputlisting{../src/main.html}

\newpage

\section{Javascript Source Code}
\lstinputlisting{../src/script.js}

\newpage

\section{CSS Source Code}
\lstinputlisting{../src/style.css}

\end{document}

