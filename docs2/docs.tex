\documentclass[letterpaper]{article}

\usepackage[utf8]{inputenc}
\usepackage[margin=1in]{geometry}
\usepackage{listings}
\usepackage[hidelinks]{hyperref}

\hypersetup{colorlinks, allcolors=blue}
\lstset {%
    breaklines=true,
    postbreak=\mbox{\textcolor{red}{$\hookrightarrow$}\space},
    basicstyle=\ttfamily,
    numbers=left,
    numberstyle=\normalsize,
    numbersep=10pt,
    frame=single,
}
\setlength\parindent{0pt}

\begin{document}

\pagenumbering{gobble}
\vspace*{\fill}
\begin{center}
    \Large
    Documentation for Financial Transactions HTML Page

    \large
    Jason N.

    June 1, 2020
\end{center}
\vspace*{\fill}

\newpage
\pagenumbering{roman}
\tableofcontents

\newpage
\pagenumbering{arabic}
\parskip 10pt

\section{HTML}\label{HTML}

\subsection{Preamble and head}

\subsubsection{Preamble}

Declares the document as HTML5.
\begin{lstlisting}[firstnumber=1]
<!DOCTYPE html>
\end{lstlisting}

\subsubsection{meta charset}

Specifies that characters in the file are encoded in UTF-8.
\begin{lstlisting}[firstnumber=4]
<meta charset = "UTF-8"/>
\end{lstlisting}

\subsubsection{link rel="stylesheet"}

Imports the CSS file.
\begin{lstlisting}[firstnumber=5]
<link rel="stylesheet" type="text/css" href="./style.css"/>
\end{lstlisting}

\subsubsection{Scripts}

Imports the main Javascript file, responsible for the table and UI.
\begin{lstlisting}[firstnumber=7]
<script src="./script.js"></script>
\end{lstlisting}

Imports the Google API library.
\begin{lstlisting}[firstnumber=8]
<script src="https://apis.google.com/js/api.js"></script>
\end{lstlisting}

Imports other Javascript files, responsible for database management.
\begin{lstlisting}[firstnumber=9]
<script src="./googleApiScript.js"></script>
<script src="./mysqlScript.js"></script>
<script src="./localStorageScript.js"></script>
<script src="./imageFirestore.js"></script>
\end{lstlisting}

\subsection{Inputs}

Disables autocomplete which remembers past user input by default. \lstinline{return false} specifies that no POST request should be made to the server.
\begin{lstlisting}[firstnumber=20]
<form onsubmit="return false" autocomplete="off">
\end{lstlisting}

\subsubsection{Labels}

Identifies the purpose of the field to the user, allows the user to select the field by clicking the label.
This element is also used by accessibility tools to identify the field.
\begin{lstlisting}[firstnumber=22]
<label for="date">Date:</label><br/>
\end{lstlisting}

\subsubsection{Date}
\subsubsection{Text}
\subsubsection{List}
\subsubsection{Buttons}
\subsection{Table}
\subsubsection{thead}
\subsubsection{tbody}

\newpage

\section{Main Javascript}\label{JS}
\subsection{getData()}
\subsection{validate()}
\subsubsection{Check empty}
\subsubsection{Check NaN}
\subsubsection{Check date}
\subsection{generateId()}
\subsection{calculateCostBasis()}
\subsection{addTransaction()}
\subsection{deleteRow()}
\subsection{editRow()}
\subsection{saveChanges()}
\subsection{discardChanges()}
\subsection{sortTable()}

\newpage

\section{CSS}\label{CSS}

\subsection{Vertical Scrolling Table}
\subsection{Horizontal Scrolling on Overflow}\label{overflow-x}
\subsection{Miscellaneous}
\subsubsection{Sort buttons}
\subsubsection{Editing highlight}
\subsubsection{Table borders}

\newpage

\appendix
\section{HTML Source Code}
\lstinputlisting{../release/public/index.html}
\newpage

\section{Javascript Source Code}
\subsection{Main Script}
\lstinputlisting{../release/public/script.js}
\newpage
\subsection{Google API Script}
\lstinputlisting{../release/public/googleApiScript.js}
\newpage
\subsection{Firebase Script}
\lstinputlisting{../release/public/firebaseScript.js}
\newpage
\subsection{MySQL Script}
\lstinputlisting{../release/public/mysqlScript.js}
\newpage
\subsection{Local Storage Script}
\lstinputlisting{../release/public/localStorageScript.js}
\newpage
\subsection{Firestore Images Script}
\lstinputlisting{../release/public/imageFirestore.js}
\newpage

\section{CSS Source Code}
\lstinputlisting{../release/public/style.css}
\newpage

\section{Server-side NodeJs Code}
\lstinputlisting{../release/functions/index.js}

\end{document}

